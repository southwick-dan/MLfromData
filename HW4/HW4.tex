\documentclass[12pt]{article}
\usepackage{fullpage}
\usepackage{amsmath,amssymb,mathtools,xparse,graphicx,float,datetime,color,array,graphics,enumerate,tikz,pgfplots,xcolor}
\usepgfplotslibrary{statistics}
\pagestyle{empty}
\newcommand{\D}{\displaystyle}
\setlength{\textheight}{9in} \setlength{\headheight}{.2in}
\setlength{\headsep}{0in} \setlength{\topmargin}{0in}
\begin{document}
\begin{center}
CSCI 6100 Machine Learning From Data\\
Fall 2018\\
\end{center}
\begin{center}
HOMEWORK 4\\
Daniel Southwick\\
661542908\\
southd@rpi.edu
\end{center}
\vspace{.1in}

\noindent {\bf Exercise 2.4} \\\\
\indent (a) Let \begin{align*} \displaystyle
			\mathbb{X} 
			= \left[\begin{array}{cccc}
			x_{10} & x_{11} & \dots & x_{1d} \\
			x_{20} & x_{21} & \dots & x_{2d} \\
			\vdots & \vdots & \ddots & \vdots \\
			x_{{d+1}0} & x_{{d+1}1} & \dots & x_{{d+1}d} 
			\end{array}\right]
			\textbf{w} 
			= \left[\begin{array}{c}
			w_0\\w_1\\\vdots\\w_d
			\end{array}\right] 	
			\textbf{y} = 
			\left[\begin{array}{c}
			y_1\\y_2\\\vdots\\y_{d+1}
			\end{array}\right]
			\end{align*}
\indent Where $\mathbb{X}$ is a nonsingular matrix with dimension $(d+1)\times(d+1)$ whose rows represents the $d+1$ points and columns represents the individual elements of a $x$. Note that $y = [\pm1,\pm1,...,\pm1]^T$\\ 
\indent If the system $sign(\mathbb{X}w)=y$ has a solution $w$, then the perceptron can shatter $(d+1)$ points. And $\mathbb{X}$ is a non singular matrix so $\mathbb{X}^{-1}$ can be computed thus $w = \mathbb{X}^{-1}y$. A solution of $w$ is found, so $d_{vc} \geq d + 1$.\\\\
\indent (b) Same as part (a), we consider each points of $\mathbb{X}$ as $[x_0,x_1,...,x_d]$. Then any new vector should be the linear combination of the $d+1$ vector. So any $d+2$ vectors of length $d+1$ have to be linearly dependent.\\

\noindent {\bf Problem 2.3} \\\\
\indent (a)Positive or negative ray: For $N$ data points, the line is split by a single point into $N+1$ regions. So positive rays $m_H(N)$ is $N+1$. And same with the negative, thus the maximal number of dichotomies is $2(N+1)-2 = 2N$ (Subtracting two cases where all points are +1 or -1). So $m_H(N)=2N$. $\displaystyle m_\mathcal{H}(3) = 6 < 2^3,\ m_H(2) = 4 = 2^2$ Thus $d_{vc}=2$\\\\
\indent (b)Positive or negative intervals: For $N+1$ regions, If the interval covers the part of two end regions, then the number of dichotomies is $2N$ like part (a), if the intervals lies within the middle $N-1$ regions, then the number of dichotomies became $\displaystyle 2{N+1\choose2} = (N-1)(N-2)$. Thus the maximum dichotomies: $\displaystyle m_H(N) = (N-1)(N-2)+2N = N^2-N+2$. $\displaystyle m_H(4) = 14 < 2^4,\ m_\mathcal{H}(3) = 8 = 2^3$ Thus $d_{vc}=3$\\\\
\indent (c)Two concentric spheres in $\mathbb{R}^d$: The problem is essentially the same as finding the interval. We can map a point in $\mathbb{R}^d$ into a point $y\in \mathbb{R}$ by $y = \sqrt{x_1^2+x_2^2+...+x_n^2}$. So for any $N$ points in $\mathbb{R}^d$, we can find the corresponding $y\in \mathbb{R}$ and then it's splits to at most $N+1$ regions, so $\displaystyle m_\mathcal{H}(N)={N+1\choose 2}+1 = \frac{1}{2}N^2+\frac{1}{2}N+1$. $\displaystyle m_H(3) = 7 < 2^3,\ m_H(2) = 4 = 2^2$. Thus $d_{vc}=2$\\\\

\noindent {\bf Exercise 2.8} \\\\
\indent For any growth function that has a break point, it then can be bounded by a polynomial function of N. If not, then $m_H(N) = 2^N$. So $1+N$, $1+N+\frac{N(N-1)}{2}$, $2^N$ are possible growth functions. \\
 For $m_H(N) = 1+N$: $m_H(1) = 2=2^1 \mbox{ and } m_H(2) = 3 < 2^2 d_{vc} = 1  m_H(N) \leq \sum_{i = 0}^{1}{N\choose i} = N+1$. \\ For $m_H(N) =1+N+\frac{N(N-1)}{2}$:
$m_H(2) = 4=2^2 \mbox{ and } m_H(3) = 7 < 2^3 d_{vc} = 2 m_H(N) \leq \sum_{i = 0}^{2}{N\choose i} = \frac{N(N-1)}{2}+N+1$. \\ For $m_H(N) =2^N$: $m_H(N) = 2^N d_{vc} = \infty m_H(N) \leq \sum_{i = 0}^{N}{N\choose i} = 2^N$\\\\
\indent And $2^{\left\lfloor{\sqrt{N}}\right\rfloor}$, $m_H(N) =2^{\left\lfloor{N/2}\right\rfloor}$ and $m_H(N) =1+N+\frac{N(N-1)(N-2)}{6}$ are not possible growth functions\\
For $m_H(N) =2^{\left\lfloor{\sqrt{N}}\right\rfloor}$: $m_H(1) = 2=2^1 \mbox{ and } m_H(2) = 2 < 2^2 d_{vc} = 1 m_H(N) \nleq \sum_{i = 0}^{1}{N\choose i} = N+1$\\
For $\displaystyle m_H(N) =2^{\left\lfloor{N/2}\right\rfloor}$ : $m_H(0) = 1=2^0 \mbox{ and } m_H(1) = 1 < 2^1 d_{vc} = 0 m_H(N) \nleq \sum_{i = 0}^{0}{N\choose i} = 1$\\
For $m_H(N) =1+N+\frac{N(N-1)(N-2)}{6}$:  $m_H(1) = 2=2^1 \mbox{ and } m_H(2) = 3 < 2^2 d_{vc} = 1 m_H(N) \nleq \sum_{i = 0}^{1}{N\choose i} = N+1$\\\\

\noindent {\bf Problem 2.10} \\\\
\indent We first separate the group of $2N$ points into two groups of $N$ points, we know that for both groups, their maximal number of dichotomies is $m_H(N)$. So for the whole $2N$ points, the maximal number of dichotomies $m_H(2N)$ are at most $m_H(N) \times m_H(N) = m_H(N)^2$, thus $m_H(2N) \leq m_H(N)^2$\\\\

\noindent {\bf Problem 2.12} \\\\
\indent We know that the sampling bound $\displaystyle = \sqrt{\frac{8}{N}\ln\frac{4m_\mathcal{H}(2N)}{\delta}} = \sqrt{\frac{8}{N}\ln\frac{4((2N)^{d_{vc}}+1)}{\delta}}$.\\ Thus $\displaystyle N\geq \frac{8}{\epsilon^2}\ln\left(\frac{4((2N)^{d_{vc}}+1)}{\delta}\right)$.\\\\ Here, $d_{vc} = 10$, $\epsilon = 0.05$, $\delta = 0.05$, so: \begin{center}$\displaystyle N \geq \frac{8}{0.05^2}\ln\left(\frac{4((2N)^{10}+1)}{0.05}\right)$\\$N \approx 452957$\\
\end{center}
\indent Thus the sample size $N$ needs to be greater than 452957.

\end{document}
